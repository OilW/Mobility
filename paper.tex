\documentclass{llncs}


\begin{document}

\title{Explain urban regional function}
\author{submission }
\maketitle
\section{Introduction}


%motivation
Urban computing importance ~\cite{1}
discovering regions of different functions~\cite{3} %3 references

%explain power
Previous model %
model complex, deep model
data heterogenous 
explainability -> disadvantage 
Explaining the result is of crucial importance


%explain implementation, we need to incorporate activity and online sentiment
Our goal is to label and cluster the region.
Furthermore, provide xxx explainations, activity , focus on aspects
activity
opinion
Example (figure)


%problem

segment regions = group regions with similar functions

function = distribution of activitys and opinions

%challenge

previous research



%opinion



%second challenge: time bins



%contribution
Our paper has an outstanding result in following several aspect:
\begin{itemize}
  \item According to what I have learnt,we are the first one to ap
  \item 2
  \item 3
\end{itemize}



%paper structure
The structure of our paper is listed as follow.Some previous work of former scholars are listed in Section 2.


%Problem and Objective.Methods and Model.Contributions and Findings

%paper structure
The remainder of the paper is organized as follows. 
We give an overview for related work in Section 2.
In Section 3, we introduce our novel models as well as proving their logic.
Section 4 presents the suprior result in experiments of our models.
Finally, we made a conclusion and looked forward to our future work in Section 5.

\section{Related Work}

Two lines of work are related to this paper: sentiment analysis into human mobility,content from social media also provide great help for us. Many scholars have made fundamental contributions,and we combined them to find interesting patterns and extended the application .

To the best of our knowledges,we are the first to combined the sentiment into human mobility.

%related work section name, find another suitable name for mobility mining
\subsection{Urban Computing}
%
Urban computing~\cite{1} tackles the major issues that cities face by analyzing human mobility collected from different sensors.  
Major sources of human mobility data are checkins in POI~\cite{}, pick-up and drop-off behavior of taxicabs~\cite{} in different locations and  trajectories~\cite{}.
%extract the different among these data
For checkins data, the most commonly adopted model is.
For taxicabs data, to fully utilize pick-up and drop-off, xxx is adopted to enhance xxx
For trajectory data, as it involves multiple points 

%tasks
The aim of mobility mining is to uncover informative patterns of improve incomes of taxi drivers~\cite{}, attracts an increasing research interest~\cite{}. 
However, existing urban computing systems extensively rely on complex machine learning algorithms hence they act as blank-boxes for end users. 
The lack of explainability weakens the persuasiveness and trustworthiness of the system for users
Our work is to provide intuitive explanations of the results for users or system designers
%most of them adopt a topic model, treating mobility patterns as words~\cite{}%topic models



\subsection{Geographical Analysis of Online Sentiment}
Recently, an emrging research interest is witnessed in exploring the geographical factors that affect online sentiment.
Empirical studies have been conducted on large-scale human mobility data, such as checkins~\cite{}, trajectory, xxx~\cite{}
Associations are found between online sentiments and geographical factors, e.g happy regions are more likely to connect with each other~\cite{milan15},a high check-in density region usually presents a more positive moode~\cite{}, 

%complete survey
However, most existing work of this line employ simple statististical analysis to uncover the associations. 
Such a coarse-grained analysis is distorted by latent variables, such as activity of the region.
Our work is the first to incorporate activity to obtain a fine-grained analysis.


\section{Application}
Our models have a wide range of application and their value

\subsection{Billboard}
Billboard

\subsection{Trajectory}

\section{Conclusion}
In this paper , we propose several novel models to d
The models have improve some extra recognition accuracy ,which have an extra contribution for functional city.


\section{Reference}
\begin{thebibliography}{1}
\bibitem{1}Zheng Y, Capra L, Wolfson O, et al. Urban Computing:Concepts, Methodologies, and Applications[J]. Acm Transactions on Intelligent Systems \& Technology, 2014, 5(3):1-55.
\bibitem{2}Zheng Y, Liu Y, Yuan J, et al. Urban computing with taxicabs[C]// International Conference on Ubiquitous Computing. ACM, 2011:89-98.
\bibitem{3}Yuan J, Zheng Y, Xie X. Discovering regions of different functions in a city using human mobility and POIs[C]// ACM SIGKDD International Conference on Knowledge Discovery and Data Mining. ACM, 2012:186-194.
\bibitem{4}%Gonz��lez M C, Hidalgo C A, Barab��si A. Understanding individual human mobility patterns[J]. Nature, 2008, 453(7196):779-782.
\bibitem{four11}Anastasios N, Salvatore S, Cecilia M, and Massimiliano P. An empirical study of geographic user activity patterns in foursquare. In Proceedings of the 5th Int��l AAAI Conference on Weblogs and Social Media, 2011.
\bibitem{milan15}Alshamsi A, Awad E, Almehrezi M, et al. Misery loves company: happiness and communication in the city[J]. Epj Data Science, 2015, 4(1):7.
\bibitem{happier16}Gallegos L, Huang A, Huang A, et al. Geography of Emotion: Where in a City are People Happier?[C]// International Conference Companion on World Wide Web. International World Wide Web Conferences Steering Committee, 2016:569-574.
\bibitem{WKB13}Warriner A B, Kuperman V, Brysbaert M. Norms of valence, arousal, and dominance for 13,915 English lemmas.[J]. Behavior Research Methods, 2013, 45(4):1191-1207.
\bibitem{SentiStrength12}Thelwall M, Buckley K, Paltoglou G. Sentiment strength detection for the social web[J]. Journal of the Association for Information Science \& Technology, 2012, 63(1):163�C173.
\bibitem{network10}Cranshaw J, Toch E, Hong J, et al. Bridging the gap between physical location and online social networks[C]// ACM International Conference on Ubiquitous Computing. ACM, 2010:119-128.
\bibitem{11}
\bibitem{12}
\bibitem{13}Wei W, Joseph K, Liu H, et al. Exploring characteristics of suspended users and network stability on Twitter[J]. Social Network Analysis \& Mining, 2016, 6(1):1-18.
\bibitem{14}Zhao S, King I, Lyu M R. A Survey of Point-of-interest Recommendation in Location-based Social Networks[J]. 2016.
\iffalse
\bibitem{15}
\bibitem{16}
\bibitem{17}
\bibitem{18}
\bibitem{19}
\bibitem{20}
\bibitem{21}
\bibitem{22}
\bibitem{23}
\bibitem{24}
\bibitem{25}
\fi
\end{thebibliography}


\end{document}